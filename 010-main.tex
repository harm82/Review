
\chapter[Working title: Calculations]
{Working title: Calculations\label{ch1}}


\chapauth{Martin R. Hediger$^{a}$, Harm Otten$^{b}$
\chapaff{$^{a}$Z\"urich\\
$^{b}$K{\o}benhavn\\
ma.hed@bluewin.ch\\
harm82@gmail.com}}

\section{Motivation}\label{sec:mot}

Let us imagine two companies A and B.
Both companies use very similar technical equipment to carry out a biotechnological process where a chemical reaction is catalyzed by an enzyme.
Company A uses an enzyme with a rate constant $k_\text{A} = 1000s^{-1}$ while company B uses an enzyme with $k_\text{B} = 2000s^{-1}$.
Letting all other things be equal, the process of company B will therefore only require half the time to produce one Mole of product compared to the time required for company A.
Company B therefore can save energy required to heat up the reaction volume, the commercial implications of this are immediate.\\
The need for efficient catalysts arises from such an outline.\footnote{We use the terms \textit{enzme} and \textit{bio-}/\textit{catalyst} interchangeably.}
Increasing the performance of enzymes however is still far from trivial and forms a growing body of research.
What is clear though is that the development of such catalysts is costly, in terms of manpower, material and energy -- if it is carried out in the laboratory.
A number of companies have in fact formed around this demand: Novozymes (DK), Genzyme (US) or DSM (NL) to name but a few\citep{meyer2013use}.\\
The laboratory costs can however be saved to a large part if the development is carried out \textit{in silico}.
The proof that computational results are as reliable as experimental results has been provided not too long ago\citep{claeyssens2006high}.




\section{Introduction}\label{sec:intro}
We provide an introduction into the topic of enzyme catalysis modeling for interested people from inside and outside of the field.




\section{Methods}\label{sec:methods}
A variety of methods has been established for the use to model enzyme catalysis.
Depending on the properties of interest, the system is treated differently.
Molecular mechanics methods allow to study the structural behavior of the enzyme over a significant time period and provide details about rearrangement of loop motives.
The description of chemical reactivity however requires the description of the electronic structure of the system (molecular mechanics models do not allow the description of bond formation and braking processes).
In this regime, again a number of specialized methods have become available.




\section{Applications}\label{sec:apps}




\section{Outlook}\label{sec:out}
